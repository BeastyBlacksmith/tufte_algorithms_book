\chapter{Introduction}
\label{ch:introduction}

Welcome to the book template!
This template is based on our previous book.\cite{Kochenderfer2019}

\section{Using Julia}

In this section we provide some examples of how Julia algorithms can be typeset, tested, etc.
We begin with a Julia algorithm, \cref{alg:sample_function}.
We use our custom \texttt{algorithm} environment with a pythontex \texttt{juliaverbatim} environment inside to typeset the algorithm.
A caption provides some additional information, and shows up in the margin.

\begin{algorithm}
\begin{juliaverbatim}
function sample_function(x, a)
	if x > a
		return log(x)
	else
		return x + log(a) - a
	end
end
\end{juliaverbatim}
\caption{
	\label{alg:sample_function}
	A sample function that takes in an evaluation scalar \jlv{x} and a scalar parameter \jlv{a}.
}
\end{algorithm}

We can add a test for our algorithm in the source code.
This test does not show up when you compile.
You can run all tests by executing \texttt{jl runtests.jl}.

\begin{juliatest}
let
	@test isapprox(sample_function(1.0, 2), -0.30685, atol=1e-5)
	@test isapprox(sample_function(1.0, 3), -0.90139, atol=1e-5)
	@test isapprox(sample_function(1.5, 2), 0.19315, atol=1e-5)
end
\end{juliatest}

The code can be executed when creating figures.
\Cref{fig:sample_function} is a standard inline figure that executes Julia code using \jlpkg{PGFPlots} to produce a \TeX file in the \texttt{\\fig} directory.
This file is then compiled into the PDF.

\begin{figure}
	\begin{jlcode}
	p = let
		xs = collect(range(0.0,stop=10.0,length=101))
		plots = Plots.Plot[]
		for a in [1,2,3,5]
			push!(plots,
				Plots.Linear(xs, [sample_function(x, a) for x in xs],
				   style="solid, thick, mark=none", legendentry="parameter \$a = $a\$"))
		end
		Axis(plots, width="8cm", height="5cm", xlabel=L"x", ylabel="sample function output",
					style="cycle list name = pastelcolors",
		 		    legendPos="outer north east")
	end
	plot(p)
	\end{jlcode}
	\begin{center}
		\plot{fig/sample_function}
	\end{center}
	\caption{
		\label{fig:sample_function}
		Curves obtained when using \jlv{sample_function} for several different values of \jlv{a}.
	}
\end{figure}

\pagebreak

Of course, there is nothing stopping you from inserting tikzpictures directly, as shown in \cref{fig:sample_tikzpicture}.
\begin{figure}
	\centering
	\begin{tikzpicture}[x=1cm, y=1cm]
		\draw[pastelBlue] (0,0) rectangle (2,2);
		\fill[pastelRed]  (3,-1) rectangle ++(2,1);
		\fill[pastelPurple]  (7,0.5) circle (0.5);
		\node[anchor=south] at (4,0) {red rectangle};
	\end{tikzpicture}
	\caption{
		\label{fig:sample_tikzpicture}
		A figure made using a \texttt{tikzpicture} environment.
	}
\end{figure}

Figures can also be placed in the margins, as with \cref{fig:sample_marginfigure}.
\begin{marginfigure}
	\begin{jlcode}
	p = let
		Axis(Plots.Linear(x->sin(x) + x, (0,10), style="solid, pastelBlue, mark=none"),
			 width="5cm", height="5cm", xlabel=L"x", ylabel=L"f(x)")
	end
	plot(p)
	\end{jlcode}
	\begin{center}
		\plot{fig/sample_marginfigure}
	\end{center}
	\caption{
		\label{fig:sample_marginfigure}
		A \texttt{marginfigure} shows up in the margin.
	}
\end{marginfigure}


The \texttt{pythontex} package also supports the \texttt{juliaconsole} environment, as shown below.
These environments act like the julia REPL, showing code as it is entered and executed.
\begin{juliaconsole}
x = 5
y = 7.5
x^y
\end{juliaconsole}

\section{Custom Environments}

We created two custom environments in addition to the \texttt{algorithm} environment when writing \textit{Algorithms for Optimization}.
The \texttt{example} environment creates an isolated section in a light-gray box with its own caption.

\begin{example}
	An example of the \texttt{example} environment.
	Place anything you want here, including figures.
	\caption{
		\label{ex:sample_example}
		An example \texttt{example}.
	}
\end{example}

We also have \texttt{question} and \texttt{solution} environments, typically used at the end of each chapter.
The solutions can automatically be placed at the end of the book by using \texttt{solutionappendix}.

\begin{question}
	What is the meaning of life?
\end{question}
\begin{solution}
	The meaning of life is \num{42}.
\end{solution}

\section{Math Notation}

\textit{Algorithms for Optimization} uses lowercase characters with bold face for vectors, such as $\vect x$, and uppercase characters with bold face for matrices, such as $\mat \Sigma$.
Scalars are lowercase with normal face, such as $\lambda$.

The contents of vectors and matrices can be typeset using \texttt{bmatrix}:
\begin{equation}
	\begin{bmatrix}x_1 \\ x_2\end{bmatrix} = \\
	\begin{bmatrix}
		\Sigma_{11} & \Sigma_{12} \\
		\Sigma_{21} & \Sigma_{22}
	\end{bmatrix}^{-1}
	\begin{bmatrix}y_1 \\ y_2\end{bmatrix}
\end{equation}

We often write out vectors horizontally in order to save space.
In these cases, we use comma-separate entries:
\begin{equation}
	\vect x = [x_1, x_2]
\end{equation}

We construct an optimization problem as follows:
\begin{equation}
	\begin{aligned}
        \minimize_{\vect x} & & f(\vect x)\\
        \subjectto & & g(\vect x) \leq \vect 0 \\
        		   & & h(\vect x) = \vect 0
    \end{aligned}
\end{equation}

See \texttt{tufte\_algorithm\_book.cls} for some additional math utilities, such as $\gaussian{\mu}{\nu}$ and $\card{\mathcal{A}}$.

\section{Conclusion}

That is more or less everything you need in order to get started writing your own textbook with the \texttt{tufte\_algorithm\_book} style.
We hope you find it useful.

If you have any questions or find improvements, please create issues on our GitHub repo: \url{https://github.com/sisl/textbook_template}.